\documentclass[a0]{sciposter}
\usepackage{lipsum}
\usepackage{graphicx}
\usepackage{amsmath}
\usepackage{booktabs}
\usepackage{listings}
\usepackage{xcolor}

% Define code style
\lstset{
    language=Python,
    basicstyle=\ttfamily\small,
    keywordstyle=\color{blue},
    stringstyle=\color{red},
    commentstyle=\color{green!60!black},
    numbers=left,
    numberstyle=\tiny,
    frame=single
}

% Define colors
\definecolor{MainColor}{RGB}{70,130,180} % Steel blue
\definecolor{SecondColor}{RGB}{176,196,222} % Light steel blue

\title{\huge SnapML: A Visual Programming Tool for Machine Learning Development}
\author{Author Name}
\institute{Institution Name}
\date{\today}
\conference{CS5351 Group Project, 2025}

\begin{document}
\maketitle

\begin{multicols}{3}
    \section{Introduction}
    \begin{itemize}
        \item SnapML is a visual programming tool designed to simplify machine learning code development
        \item It transforms complex coding processes into an intuitive drag-and-drop interface
        \item Supports real-time code generation and preview
    \end{itemize}
    
    \section{System Architecture}
    % Add system architecture diagram and description here
    The SnapML system consists of:
    \begin{itemize}
        \item Frontend interface with drag-and-drop components
        \item Backend code generation engine
        \item Integration with multiple ML frameworks
        \item Data visualization tools
    \end{itemize}
    
    \section{Key Features}
    \begin{itemize}
        \item Visual programming interface
        \item Support for multiple machine learning frameworks
        \item Real-time code generation
        \item Data visualization
        \item Model evaluation tools
    \end{itemize}
    
    \section{Case Study: Iris Classification}
    % Add workflow diagram and code example
    \begin{lstlisting}
import seaborn as sns
from pycaret.classification import *
from sklearn.model_selection import train_test_split
import pandas as pd

# Load data
iris_data = sns.load_dataset("iris")
setup(iris_data, target = 'species')

# Split data
train_X, test_X, train_Y, test_Y = train_test_split(
    iris_data.drop(columns = ['species']),
    iris_data['species'], 
    test_size=0.1, 
    random_state=42)
    
# Create and evaluate model
RandomForest_ML = create_model('rf')
output = predict_model(RandomForest_ML, data=test_X)
    \end{lstlisting}
    
    \section{Results}
    % Add performance metrics, charts, etc.
    The SnapML platform enables:
    \begin{itemize}
        \item Rapid prototyping of ML models
        \item Increased productivity for data scientists
        \item Better collaboration between technical and non-technical team members
        \item Simplified educational experience for ML beginners
    \end{itemize}
    
    \section{Conclusion \& Future Work}
    \begin{itemize}
        \item SnapML provides an intuitive interface for machine learning development
        \item Future work will expand framework support and add more advanced visualization tools
        \item Planned integration with cloud deployment platforms
        \item Community-driven expansion of component library
    \end{itemize}
\end{multicols}

\end{document} 